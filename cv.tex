%%%%%%%%%%%%%%%%%%%%%%%%%%%%%%%%%%%%%%%%%
% Medium Length Professional CV
% LaTeX Template
% Version 2.0 (8/5/13)
%
% This template has been downloaded from:
% http://www.LaTeXTemplates.com
%
% Original author:
% Trey Hunner (http://www.treyhunner.com/)
% Heavy modification by Joseph Eatson
%
% Important note:
% This template requires the resume.cls file to be in the same directory as the
% .tex file. The resume.cls file provides the resume style used for structuring the
% document.
%
%%%%%%%%%%%%%%%%%%%%%%%%%%%%%%%%%%%%%%%%%

\documentclass{resume} % Use the custom resume.cls style
\usepackage[a4paper,left=0.75in,top=0.6in,right=0.75in,bottom=0.6in]{geometry} % Document margins
\usepackage{hyperref}
\hypersetup{
    colorlinks=true,
    linkcolor=blue,
    filecolor=magenta,      
    urlcolor=cyan,
    pdftitle={Joseph Eatson CV},
    pdfpagemode=FullScreen,
    }

% Personal details
\name{Joseph Eatson} % Your name
\address{17 Stanmore Avenue \\ Leeds \\ United Kingdom \\ LS4 2RP}
\address{\texttt{\href{mailto:py13je@leeds.ac.uk}{py13je@leeds.ac.uk}} \\ \texttt{\href{mailto:jweatson@gmail.com}{jweatson@gmail.com}} \\ They/He}

% Begin document
\begin{document}
% Education section, covering university and secondary education down to GCSE level
\begin{rSection}{Education}

{\bf University of Leeds} \dotfill {\sl 2013-2022} \\ 
Ph.D. in Astrophysics - {\em Numerical Simulations of Dusty Colliding Wind Binaries} \dotfill In progress \\
MPhys in Physics \& Astrophysics \dotfill 2:1 \\
BSc in Physics \dotfill 2:1

{\bf Enfield Grammar School} \dotfill {\sl 2006-2013} \\
A Level \dotfill A in History \& Mathematics, B in Physics \\
GCSE \dotfill 13 graded C or higher, with 6 graded A

\end{rSection}

% Detail research projects, link to work when applicable
\begin{rSection}{Research Projects}

\begin{rSubsection}{Numerical Simulations of Dusty Colliding Wind Binaries}{\href{https://raw.githubusercontent.com/atomsite/Thesis/master/Thesis.pdf}{Thesis - in progress}}{Ph.D. Research Project - University of Leeds}{2017-2022}
\item Project centred around the creation of a highly performant numerical code for performing analysis of dust formation in Colliding Wind Binary systems.
\item Performed extensive modification to existing Athena++ and MG hydrodynamical codes to achieve this goal, co-ordination with developers of both projects, as well as general debugging and reporting.
\item Performed parameter space exploration on requirements for dust formation in Colliding Wind Binary Systems, varying wind momentum ratio, cooling parameters and separation distance.
\item Simulations on observed systems such as WR140 and $\eta$ Carinae performed, with particular focus on the impact of orbital eccentricity on dust formation rates.
\item During this time developed a novel passive scalar model for simulating dust growth, destruction and cooling within a numerical simulation. Model is highly extensible and potentially applicable to a range of other hydrodynamical codes.
\end{rSubsection}

\begin{rSubsection}{\textit{A Comedy of Uncertainties} - Mapping Stellar Clusters Using Spatial \& Multi-Stage Sub-Clustering Methods}{\href{https://raw.githubusercontent.com/atomsite/masters-project/main/masters-report.pdf}{Project Report}}{MPhys Research Project - University of Leeds}{2016-2017}
\item Experimentation with sub-clustering methods for application in open clusters and OB associations.
\item Used the R statistical language to perform sub-clustering, provisional parallax data from Hipparcos-Gaia catalog was used to map stellar clusters in 3D, with the long-term goal of resolving kinematics.
\item Initial results were promising, but conclusive results were dependent on 2\textsuperscript{nd} Gaia data release that was not available until a year after project submission.
\end{rSubsection}

\end{rSection}

% Detail skills, mainly computing and research
\begin{rSection}{Skills}

\begin{rPoints}{Programming}
	\item Significant experience in several programming languages, from statistical languages such as Python, R and Julia, to low-level systems development languages such as C, C++, Fortran90 and Rust.
	\item Particularly fluent in C, C++, Python and R, having more than 6 years of daily usage of each language.
	\item Writes lean, well-documented code on-time with an emphasis on readability and parallel performance.
	\item Experience in modern development techniques and version control systems such as Git and Mercurial.
	\item Ph.D. required the understanding of HPC concepts such as shared memory and message passing parallelism, in particular the OpenMPI and OpenMP libraries.
	\item Familiar with other HPC concepts such as GPGPU accelerators, and have written programmes using CUDA for personal projects.
	\item Daily usage of IDEs such as Spyder, Rstudio, JuPyter and VSCode.
\end{rPoints}

\begin{rPoints}{Research Skills}
	\item High degree of knowledge in writing academic papers for peer review.
	\item Familiarity with modern documentation and static analysis methods such as Doxygen.
	\item Very proficient in the \LaTeX \,typesetting language as well as the Bib\TeX\, and Bib\LaTeX \,citation formats.
	\item Quick study for new concepts and technical jargon.
	\item Postgraduate level background in physics \& mathematics, IOP-accredited degree in astrophysics, with additional knowledge in numerical methods, fluid dynamics, quantum computing and computer science.
\end{rPoints}

\begin{rPoints}{Computing}
	\item Extreme proficiency with all operating systems and desktop environments, with years of experience in Windows, MacOS \& multiple Linux distributions.
	\item Personal experience in server maintenance and network management, as well as general technical support, typically the first point of call for most people in my department with a computing issue.
	\item Proficiency in office suites such as Microsoft Office, iWork and Google Workspace, as well as their open source counterparts.
\end{rPoints}

\begin{rPoints}{Teaching \& Collaboration}
	\item Experience teaching students in a wide range of age groups, from primary school to university level.
	\item Able to write, explain and defend concepts clearly and concisely to audiences ranging from students to seniors, educators to executives.
	\item Proficient in teams of any size, I work well with others, and above all else pride myself in being an asset to those that I work with.
\end{rPoints}

\end{rSection}

% Skills section, but in brief, cover additional things that don't require a bullet point
\begin{rSection}{Overview}

{\bf Teaching} \dotfill 5 years teaching \& assessing lab skills and Python to undergraduates \\
{\bf Fluent Programming Languages} \dotfill C, C++, Python 2.7-3.9, R \\
{\bf Additional Programming Languages} \dotfill Fortran90, Julia, Rust \\
{\bf Libraries \& APIs} \dotfill OpenMP, OpenMPI, Numba, Cython \\
{\bf Practical Knowledge} \dotfill Telescope operation, server maintenance \\
{\bf Tools \& Environments} \dotfill VSCode, JuPyter, RStudio, GNUPlot, Athena++, SGE, \LaTeX \\
{\bf Programming Strengths} \dotfill Highly-optimised, multi-threaded code for use in HPC environments

\end{rSection}

% Final section on references
\begin{rSection}{References}
	{\bf Dr. Julian Pittard} \dotfill {\sl Ph.D. supervisor - University of Leeds} \\
	\null\dotfill 0113 343 3805, \texttt{\href{mailto:J.M.Pittard@leeds.ac.uk}{J.M.Pittard@leeds.ac.uk}} \\ 
	{\bf Dr. Stuart Lumsden} \dotfill {\sl Masters project supervisor - University of Leeds} \\
	\null\dotfill 0113 343 6691, \texttt{\href{mailto:S.L.Lumsden@leeds.ac.uk}{S.L.Lumsden@leeds.ac.uk}} \\ 
%	{\bf Mr. Charles Irvine} \dotfill {\sl Previous employer - Questions of Difference Ltd.} \\
%	\null\dotfill \texttt{\href{mailto:charles@questionsofdifference.com}{charles@questionsofdifference.com}} \\ 
\end{rSection}

% Finish up, hopefully this compiles!
\end{document}
