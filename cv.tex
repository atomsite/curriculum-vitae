%%%%%%%%%%%%%%%%%%%%%%%%%%%%%%%%%%%%%%%%%
% Medium Length Professional CV
% LaTeX Template
% Version 2.0 (8/5/13)
%
% This template has been downloaded from:
% http://www.LaTeXTemplates.com
%
% Original author:
% Trey Hunner (http://www.treyhunner.com/)
%
% Important note:
% This template requires the resume.cls file to be in the same directory as the
% .tex file. The resume.cls file provides the resume style used for structuring the
% document.
%
%%%%%%%%%%%%%%%%%%%%%%%%%%%%%%%%%%%%%%%%%

%----------------------------------------------------------------------------------------
%	PACKAGES AND OTHER DOCUMENT CONFIGURATIONS
%----------------------------------------------------------------------------------------

\documentclass{resume} % Use the custom resume.cls style

\usepackage[left=0.75in,top=0.6in,right=0.75in,bottom=0.6in]{geometry} % Document margins

\usepackage{hyperref}
\hypersetup{
    colorlinks=true,
    linkcolor=blue,
    filecolor=magenta,      
    urlcolor=cyan,
    pdftitle={Overleaf Example},
    pdfpagemode=FullScreen,
    }

\name{Joseph Eatson} % Your name
\address{17 Stanmore Avenue \\ Leeds \\ West Yorkshire \\ United Kingdom \\ LS4 2RP}
\address{\href{mailto:py13je@leeds.ac.uk}{py13je@leeds.ac.uk} \\ \href{mailto:jweatson@gmail.com}{jweatson@gmail.com} \\ they/he}
\begin{document}

%----------------------------------------------------------------------------------------
%	EDUCATION SECTION
%----------------------------------------------------------------------------------------

\begin{rSection}{Education}

{\bf University of Leeds} \dotfill {\bf \em 2013-2022} \\ 
Ph.D. in Astrophysics - {\em Numerical Simulations of Dusty Colliding Wind Binaries} \dotfill In Progress \\
BSc \& MPhys in Physics \& Astrophysics \dotfill 2:1

{\bf Enfield Grammar School} \dotfill {\bf \em 2006-2013} \\
A-Levels \dotfill A in History \& Mathematics, B in Physics \\
GCSEs \dotfill 13, with 6 A-grade

\end{rSection}

%----------------------------------------------------------------------------------------
%	WORK EXPERIENCE SECTION
%----------------------------------------------------------------------------------------

\begin{rSection}{Research Projects}

\begin{rSubsection}{Numerical Simulations of Dusty Colliding Wind Binaries}{\href{https://raw.githubusercontent.com/atomsite/Thesis/master/Thesis.pdf}{Thesis}}{Ph.D. Research Project - University of Leeds}{2017-2022}
\item Creation of highly performant numerical code for performing fluid dynamics simulations of Colliding Wind Binary systems.
\item Extensive modification to existing Athena++ hydrodynamical code.
\item Performed parameter space exploration on requirements for dust formation in Colliding Wind Binary Systems.
\item Simulations on observed systems such as WR98a, WR104 and WR140 performed, with particular interest in impact of orbital eccentricity on dust formation rates. 
\item Novel passive scalar model for simulating dust growth, destruction and cooling within a numerical simulation.
\end{rSubsection}

\begin{rSubsection}{A Comedy of Uncertainties - Mapping Stellar Clusters Using Spatial \& Multi-Stage Sub-Clustering Methods}{\href{https://raw.githubusercontent.com/atomsite/masters-project/main/masters-report.pdf}{Project Report}}{MPhys Research Project - University of Leeds}{2016-2017}
\item Experimentation with sub-clustering methods for application in open clusters and OB associations.
\item Used R statistical language to perform sub-clustering.
\item Results were promising, but subject to additional data from GAIA satellite that was not available until after submission.
\end{rSubsection}


\end{rSection}


%----------------------------------------------------------------------------------------
%	TECHNICAL STRENGTHS SECTION
%----------------------------------------------------------------------------------------

\begin{rSection}{Skills}


{\bf Teaching} \dotfill 5 years teaching \& assessing lab skills and Python to undergraduates \\
{\bf Fluent Programming Languages} \dotfill C, C++, Python 2.7-3.9, R \\
{\bf Additional Programming Languages} \dotfill Fortran90, Julia, Rust \\
{\bf Libraries \& APIs} \dotfill OpenMP, OpenMPI, Numba, Cython \\
{\bf Practical Knowledge} \dotfill Telescope operation, server \& \\
{\bf Tools \& Environments} \dotfill VSCode, RStudio, GNUPlot, Athena++, SGE, \LaTeX \\
{\bf Programming Strengths} \dotfill Highly-optimised, multi-threaded code for use in HPC environments

\end{rSection}

%----------------------------------------------------------------------------------------
%	EXAMPLE SECTION
%----------------------------------------------------------------------------------------

%\begin{rSection}{Section Name}

%Section content\ldots

%\end{rSection}

%----------------------------------------------------------------------------------------

\end{document}
